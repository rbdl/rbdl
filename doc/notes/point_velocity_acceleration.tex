%        File: computation_notes.tex
%     Created: Mon Dec 20 02:00 PM 2010 C
% Last Change: Mon Dec 20 02:00 PM 2010 C
%
\documentclass[a4paper]{article}
\usepackage[utf8]{inputenc}
\usepackage{graphicx}

\newcommand{\Spa}[1]{\mathbf{\hat{#1}}}
\newcommand{\Nspa}[1]{\mathbf{\underline{#1}}}
\newcommand{\Vec}[1]{\mathbf{#1}}

\begin{document}

\section{Computation of the velocity of a point on a rigid body using spatial
algebra}

The ABA computes automatically the spatial velocity of body $i$ in the
reference frame of the body such that computation of joint velocities are
easier. Now we want to compute the (linear) velocity in base coordinates of a
point $\Vec{p}$ that is attached to body $i$ with body coordinates
${}^i\Vec{r}_p$, i.e. we want to compute ${}^0 \Vec{\dot{r}}_p = {}^0
\Vec{r}_p$.

First of all we compute the spatial velocities ${}^0\Spa{v}$ of all bodies in
base coordinates (which can be done during the first loop of the ABA):

\begin{equation}
	{}^0\Spa{v}_i = {}^0\Spa{v}_{\lambda(i)} + {}^{0}\Spa{X}_{i} \Spa{v}_{Ji}
\end{equation}

where $\Spa{v}_{Ji} = \Spa{S}_i \dot{q}_i $ which is the velocity that is
propagated by joint $i$ from body $\lambda(i)$ to body $i$ along joint axis
$\Spa{S}_i$ (see also RBDA p. 80).

This is now the velocity of the body $i$ in base coordinates. What is now left
todo is to transform this velocity into the velocity of the point $\Vec{p}$.
Before we can do that we have to compute the coordinates of $P$ in base
coordinates which can be done by:

\begin{equation}
	{}^0\Vec{r}_p = {}^0 \Vec{r}_i + {}^0\Vec{R}_i {}^i \Vec{r}_p
\end{equation}

for which ${}^0 \Vec{r}_i$ is the origin of the bodies in base coordinates and
${}^0 \Vec{R}_i$ the orientation of the base relative to the body.

Now we can compute the velocity of point $\Vec{r}_p$ with the following
formulation:
\begin{equation}
	{}^0\Spa{v}_p = \textit{xlt}({}^0\Vec{r}_p) {}^0\Spa{v}_i =
	\left[
	\begin{array}{cc}
		\Vec{1} & 0 \\
		-{}^0\Vec{r}_p \times & \Vec{1}
	\end{array}
	\right]
	{}^0\Spa{v}_i
\end{equation}

By doing so, the linear part of the spatial velocity ${}^0\Spa{v}_p$ has the
following entries:
\begin{equation}
	{}^0\Spa{v}_p = \left[
	\begin{array}{c}
		\Nspa{\omega} \\
		-{}^0\Nspa{r}_p \times \Nspa{\omega} + {}^0 \Nspa{\Vec{v}}_i
	\end{array}
	\right]
	=
	\left[
	\begin{array}{c}
		\Nspa{\omega} \\
		{}^0 \Nspa{\Vec{v}}_i + \Nspa{\omega} \times {}^0 \Nspa{r}_p
	\end{array}
	\right]
\end{equation}

For which the bottom line is the term for linear velocity in the standard 3D
notation.

\section{Computation of the acceleration of a point on a rigid body using
spatial algebra}

The acceleration of a point depends on three quantities: the position of the
point, velocity of the body and the acceleration of the body. We therefore
assume that we already computed the velocity as described in the previous
section.

To compute the acceleration of a point we have to compute the spatial acceleration of a
body in base coordinates, similar to the velocity computation:
\begin{equation}
	{}^0\Spa{a}_i = {}^0\Spa{a}_{\lambda(i)} + {}^{0}\Spa{X}_{i} \Spa{a}_{Ji}
\end{equation}
for which $\Spa{a}_{Ji} = \Spa{S}_i \ddot{q}_i$.

With this we can compute the acceleration of point $\Vec{p}$ by:

\begin{eqnarray}
	{}^0\Spa{a}_p & = &{}^0\Spa{v}_i \times^ * \left( xlt\left({}^0\Vec{r}_p\right)
	{}^0\Spa{v}_i \right) + xlt({}^0\Vec{r}_p) {}^0\Spa{a}_i \\
	& = & \left[
	\begin{array}{cc}
		\Nspa{\omega}_i \times & \Nspa{v}_i \times \\
		\Vec{0} & \Nspa{\omega}_i \times
	\end{array}
	\right]
	\left[
	\begin{array}{cc}
		\Vec{1} & \Vec{0} \\
		-\Nspa{r}_p \times & \Vec{1}
	\end{array}
	\right]
	\left[
	\begin{array}{c}
		\Nspa{\omega}_i \\
		\Nspa{v}_i
	\end{array}
	\right]
	+
	\left[
	\begin{array}{cc}
		\Vec{1} & \Vec{0} \\
		- \Nspa{r}_p \times & \Vec{1}
	\end{array}
	\right]
	\left[
	\begin{array}{c}
		\Nspa{\dot{\omega}}_i \\
		\Nspa{a}_i
	\end{array}
	\right]
	\\
	& = & \left[
	\begin{array}{cc}
		\Nspa{\omega}_i \times & \Nspa{v}_i \times \\
		\Vec{0} & \Nspa{\omega}_i \times
	\end{array}
	\right]
	\left[
	\begin{array}{c}
		\Nspa{\omega}_i \\
		\Nspa{v}_p
	\end{array}
	\right]
	+
	\left[
	\begin{array}{c}
		\Nspa{\dot{\omega}}_i \\
		\Nspa{a}_i + \Nspa{\dot{\omega}}_i \times \Nspa{r}_p
	\end{array}
	\right]
	\\
	& = & \left[
	\begin{array}{c}
		\Nspa{v}_i \times \Nspa{v}_p \\
		\Nspa{\omega}_i \times \Nspa{v}_p
	\end{array}
	\right]
	+
	\left[
	\begin{array}{c}
		\Nspa{\dot{\omega}}_i \\
		\Nspa{a}_i + \Nspa{\dot{\omega}}_i \times \Nspa{r}_p
	\end{array}
	\right]
\end{eqnarray}

Looking again at the components of the rotational and translational part of
the combined 6D vector reveals the following (after some re-ordering):

\begin{equation}
	{}^0\Nspa{a}_p = \Nspa{a}_i +
	\Nspa{\dot{\omega}}_i
	\times
	{}^0\Nspa{r}_p
	+
	\Nspa{\omega}_i \times \Nspa{v}_p
	\label{}
\end{equation}


We shall derive this equation with standard 3D notation. Assuming we have the
velocity of point $\Vec{p}$ given as ${}^0\Vec{v}_p = {}^0\Vec{v}_i + \Vec{\omega}_i
\times {}^0\Vec{r}_p$. Taking the derivative gives us:
\begin{eqnarray}
	\frac{d}{dt} {}^0 \Vec{v}_p & = & {}^0 \Vec{a}_i + \Vec{\dot{\omega}}_i \times
	{}^0 \Vec{r}_p + \Vec{\omega}_i \times {}^0 \Vec{ v}_p
	\label{}
\end{eqnarray}
as both $\Vec{\omega}$ and ${}^0\Vec{r}_p$ can change over time. The
expression we get is exactly what we found for the translational part of our 6D vector.

\includegraphics[width=0.9\textwidth]{acceleration_visualization}

\end{document}
